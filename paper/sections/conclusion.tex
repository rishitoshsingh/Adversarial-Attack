\section{Conclusion}

    As we move toward a future that incorporates more and more neural networks and deep learning algorithms in our daily lives we have to be careful to remember that these models can be fooled very easily. Despite the fact that neural networks are to some extent biologically inspired and have near (or super) human capabilities in a wide variety of tasks, adversarial examples teach us that their method of operation is nothing like how real biological creatures work. As we’ve seen neural networks can fail quite easily and catastrophically, in ways that are completely alien to us humans.
    We do not completely understand neural networks and to use our human intuition to describe neural networks would be unwise. For example, often times you will hear people say something to the effect of “the neural network thinks the image is of a cat because of the orange fur texture.” The thing is a neural network does not “think” in the sense that humans “think.” They are fundamentally just a series of matrix multiplications with some added non-linearities. And as adversarial examples show us, the outputs of these models are incredibly fragile. We must be careful not to attribute human qualities to neural networks despite the fact that they have human capabilities. That is, we must not anthropomorphize machine learning models.

    All in all, adversarial examples should humble us. They show us that although we have made great leaps and bounds there is still much that we do not know.
    All source code is available on GitHub (https://github.com/rishitoshsingh/Adversial-Attack)
